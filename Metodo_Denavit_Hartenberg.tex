\documentclass{article}%
\usepackage[T1]{fontenc}%
\usepackage[utf8]{inputenc}%
\usepackage{lmodern}%
\usepackage{textcomp}%
\usepackage{lastpage}%
\usepackage{geometry}%
\geometry{margin=1in}%
\usepackage{amsmath}%
%
\usepackage{amsmath}%
\usepackage{utf8x}%
\title{Método de Denavit{-}Hartenberg (DH)}%
\author{Generado por Grok 3 (xAI)}%
\date{20 de febrero de 2025}%
%
\begin{document}%
\normalsize%
\maketitle%
\section{Respuesta Directa}%
\label{sec:RespuestaDirecta}%
El método de Denavit{-}Hartenberg (DH) es una técnica estándar en robótica para describir la cinemática de brazos robóticos en serie, usando cuatro parámetros por junta.\newline%
%
Se introdujo en 1955 por Jacques Denavit y Richard Hartenberg para estandarizar sistemas de coordenadas.\newline%
%
Los parámetros son: ángulo de rotación (θ), desplazamiento a lo largo del eje z (d), longitud del enlace (a), y ángulo de torsión (α).\newline%
%
Es sorprendente cómo este método, aunque técnico, permite calcular fácilmente la posición y orientación del efector final, como en un robot plano de 2 juntas, donde la posición depende de ángulos y longitudes específicas.\newline%
%
\subsection{¿Qué es el Método DH?}%
\label{subsec:QueselMtodoDH?}%
El método DH asigna marcos de coordenadas a cada junta de un robot para modelar su movimiento. Cada junta tiene un marco con un eje z alineado con el eje de movimiento (rotación o traslación) y un eje x perpendicular, facilitando cálculos cinemáticos.\newline%

%
\subsection{¿Cómo Funciona?}%
\label{subsec:CmoFunciona?}%
Usa cuatro parámetros por junta: θ (ángulo variable para juntas rotativas), d (desplazamiento variable para juntas prismáticas), a (longitud fija del enlace), y α (ángulo fijo de torsión). Estas se combinan en matrices de transformación homogénea para encontrar la posición del efector final.\newline%

%
\subsection{Ejemplo Simple}%
\label{subsec:EjemploSimple}%
Para un robot plano de 2 juntas rotativas:\newline%
%
{-} Junta 1: θ₁ variable, d₁ = 0, a₁ fija, α₁ = 0.\newline%
%
{-} Junta 2: θ₂ variable, d₂ = 0, a₂ fija, α₂ = 0.\newline%
%
La posición del efector final es:\newline%
%
\[%
x = a_1 \cos(\theta_1) + a_2 \cos(\theta_1 + \theta_2)%
\]%
\[%
y = a_1 \sin(\theta_1) + a_2 \sin(\theta_1 + \theta_2)%
\]%
\[%
z = 0%
\]

%
\end{document}